<?xml version="1.0" encoding="UTF-8"?>

   \document{class="article"}

     

    \section{Calibration Operations}

    <description>
      This operating mode is used to perform various calibration operations at different times
      of the day and night. These include:-
      <ul>
	<li> Twilight (sky) flats.</li>
	<li> Focus procedure.</li>
	<li> Standards for various instruments.</li>
	<li> Autoguider focus.</li>
	<li> Pointing calibration.</li>
	<li> Daytime calibration for instruments.</li>
      </ul>
      The times for these operations will vary from, night to night - e.g. dependant on sunset time etc 
      and may require to be run at fixed or variable intervals in the night. 
      They may also depend on previous invokations of themselves and other calibration operations.
      
      <p/>There are 2 fundamentally different approaches to the scheduling of these ops.
	
	<p/>LT Model - This involves scheduling the ops at times relative to sunset, twilight etc - 
	  The daily variation is then taken care of automatically.
	  
	  <p/>FT Model - The times must be fitted into gaps between fixed RTI sessions which are predefined 
	    upto a month in advance - the daily variation would then have to be explicitly scheduled taking 
	    into account these RTI windows.
	    
	    <p/>The aim of this workflow is to identify ways in which both models can be incorporated into 
	      the CAL MCA configuration.
	      
	      <p/>This will require both the CAL MCAs? (DCAL and NCAL) or a combined class and the 
		individual CalibrationTasks? and subclasses to be modified and configured appropriately.
    </description><p/><hr color="blue"/><br/><span style="display:block;font-size:16pt;font-color:blue;background:#ccff22"><u>xx</u> -    CALIB/General configuration and implementation</span><p/><description>
	Covers the imlementation of the CAL MCA, its configuration and that of the various MTAs which
	implement the different CALIB operations.
      </description><p/><div style="display:block;position:relative;left:20;background:#ff22cc">
      Notes associated with this workflow.
      <ul><font color="blue"/></ul></div><p/><div style="display:block;position:relative;left:20px;width:90%;background:#cc22cc">
      Changelog for this workflow.
      <table border="1" bgcolor="lightgrey"/></div><p/><div style="display:block;position:relative;left:20px;width:90%;background:#cc22cc">
     New grouped stuff by class
     <table bgcolor="#ccddee" frame="hsides" border="1"/></div>

    \section{Science Operations}

    <description>
      Science or offline operations are performed by default in any available time. This is the default
      operating mode of the telescope during the night.
    </description><p/><hr color="blue"/><br/><span style="display:block;font-size:16pt;font-color:blue;background:#ccff22"><u/> -    SCIENCE/Autoguider integration</span><p/><description>
	Integration of the autoguider to science programs.
      </description><p/><div style="display:block;position:relative;left:20;background:#ff22cc">
      Notes associated with this workflow.
      <ul><font color="blue"/></ul></div><p/><div style="display:block;position:relative;left:20px;width:90%;background:#cc22cc">
      Changelog for this workflow.
      <table border="1" bgcolor="lightgrey"/></div><p/><div style="display:block;position:relative;left:20px;width:90%;background:#cc22cc">
     New grouped stuff by class
     <table bgcolor="#ccddee" frame="hsides" border="1"/></div><p/><hr color="blue"/><br/><span style="display:block;font-size:16pt;font-color:blue;background:#ccff22"><u/> -    SCIENCE/Group types handling</span><p/><description>
	Covering different group types and special handling.
      </description><p/><div style="display:block;position:relative;left:20;background:#ff22cc">
      Notes associated with this workflow.
      <ul><font color="blue"/></ul></div><p/><div style="display:block;position:relative;left:20px;width:90%;background:#cc22cc">
      Changelog for this workflow.
      <table border="1" bgcolor="lightgrey"/></div><p/><div style="display:block;position:relative;left:20px;width:90%;background:#cc22cc">
     New grouped stuff by class
     <table bgcolor="#ccddee" frame="hsides" border="1"/></div><p/><hr color="blue"/><br/><span style="display:block;font-size:16pt;font-color:blue;background:#ccff22"><u/> -    SCIENCE/Error Handling</span><p/><description>
		Intelligent handling of error conditions.
      </description><p/><div style="display:block;position:relative;left:20;background:#ff22cc">
      Notes associated with this workflow.
      <ul><font color="blue"><li><u>ENT</u>, 2005=02-24T08:34 -  <note class="ENT" date="2005=02-24T08:34">
		There is a problem with the LT rotator due to the way it is kept at ROT MNT 0 + FLOAT and never
		moves far from this position - it appears to 'stick' physically. A program of excercise is proposed
		<ul>
	  		<li>At the start of each slew to observation.</li>
	  		<li>At the start of the night.</li>
		</ul>
      </note></li></font></ul></div><p/><div style="display:block;position:relative;left:20px;width:90%;background:#cc22cc">
      Changelog for this workflow.
      <table border="1" bgcolor="lightgrey"><tr><td fgcolor="blue">ObservationSequenceTask</td><td fgcolor="blue">2005=02-24T08:34</td><td fgcolor="blue">
	    	A set of new variables has been added to handle the start-of-observation excercise of the rotator.
	    	A flag doExcerciseRotator is set from config parameter excercise.rotator which
	    	determines if the rotator should be excercised each slew. The variables rotatorExcerciseAngleLo
	    	and rotatorExcerciseAngleHi are set respectively from config parameters
	    	rotator.excercise.angle.lo and rotator.excercise.angle.hi. If not specified the defaults
	    	are set from the constants DEFAULT_ROTATOR_EXCERCISE_ANGLE_LO and DEFAULT_ROTATOR_EXCERCISE_ANGLE_HI.
	    	There are some changes to the task sequencing code to deal with this option - this may need more
	      	looking at as there are quite a few variations depending on whether either or both the doRotate 
	      	and doSlew options are selected - watch this.
	  	</td></tr></table></div><p/><div style="display:block;position:relative;left:20px;width:90%;background:#cc22cc">
     New grouped stuff by class
     <table bgcolor="#ccddee" frame="hsides" border="1"><tr><td valign="top"><b>ObservationSequenceTask</b><p/>ngat.rcs.science</td><td><table><tr><td valign="top">2005=02-24T08:34</td><td>
	    	A set of new variables has been added to handle the start-of-observation excercise of the rotator.
	    	A flag doExcerciseRotator is set from config parameter excercise.rotator which
	    	determines if the rotator should be excercised each slew. The variables rotatorExcerciseAngleLo
	    	and rotatorExcerciseAngleHi are set respectively from config parameters
	    	rotator.excercise.angle.lo and rotator.excercise.angle.hi. If not specified the defaults
	    	are set from the constants DEFAULT_ROTATOR_EXCERCISE_ANGLE_LO and DEFAULT_ROTATOR_EXCERCISE_ANGLE_HI.
	    	There are some changes to the task sequencing code to deal with this option - this may need more
	      	looking at as there are quite a few variations depending on whether either or both the doRotate 
	      	and doSlew options are selected - watch this.
	  	</td></tr></table></td></tr></table></div><p/><hr color="blue"/><br/><span style="display:block;font-size:16pt;font-color:blue;background:#ccff22"><u>WF_013</u> -    SCIENCE/Observation sequencing </span><p/><description>
	This workflow deals with the sequencing of operations during an observation in science mode.
      </description><p/><div style="display:block;position:relative;left:20;background:#ff22cc">
      Notes associated with this workflow.
      <ul><font color="blue"><li><u>ENT</u>, 2005-03-23T11:40 -  <note class="ENT" date="2005-03-23T11:40">
		At the Dev Mtg 22-03-2005 it was decided to institute the use of the INST ACQUIRE command.
		<p/>This envisages a procedure as follows:-
	  	<ol>
	    	<li>Slew onto target - TCS GOTO command</li>
	    	<li>Configure instrument - ICS CONFIG command</li>
	    	<li>Request instrument adjusts position - ICS ACQUIRE command</li>
	    	<li>Expose instrument - ICS EXPOSE command e.g. MULTRUN</li>	 
	  	</ol> 
	  	<p/>Note: Slew and Config can be issued in parallel, then Acquire then Expose.
      </note></li></font></ul></div><p/><div style="display:block;position:relative;left:20px;width:90%;background:#cc22cc">
      Changelog for this workflow.
      <table border="1" bgcolor="lightgrey"><tr><td fgcolor="blue">GroupSequenceTask</td><td fgcolor="blue">2005=04-04T15:00</td><td fgcolor="blue">
	    The constructor has been modified to accept a new parameter maxGroupLength
	    detailing the maximum length of valid group (nominal exec time). This is passed by SCA
	    from its own configurable maxGroupLength. The value is passed into the ScheduleTask as its
	    latestTime value.
	  </td></tr><tr><td fgcolor="blue">ScienceControlAgent</td><td fgcolor="blue">2005=04-04T15:00</td><td fgcolor="blue">
	    New variable maxGroupLength detailing the maximum length of valid group (nominal exec time).
	    This is obtained from config parameter max.group.length and set to the default value of:-
	    DEFAULT_MAX_GROUP_LENGTH if not configured. It is passed as an extra parameter to the GroupSequenceTask.
	  </td></tr></table></div><p/><div style="display:block;position:relative;left:20px;width:90%;background:#cc22cc">
     New grouped stuff by class
     <table bgcolor="#ccddee" frame="hsides" border="1"><tr><td valign="top"><b>GroupSequenceTask</b><p/>ngat.rcs.science</td><td><table><tr><td valign="top">2005=04-04T15:00</td><td>
	    The constructor has been modified to accept a new parameter maxGroupLength
	    detailing the maximum length of valid group (nominal exec time). This is passed by SCA
	    from its own configurable maxGroupLength. The value is passed into the ScheduleTask as its
	    latestTime value.
	  </td></tr></table></td></tr><tr><td valign="top"><b>ScienceControlAgent</b><p/>ngat.rcs.science</td><td><table><tr><td valign="top">2005=04-04T15:00</td><td>
	    New variable maxGroupLength detailing the maximum length of valid group (nominal exec time).
	    This is obtained from config parameter max.group.length and set to the default value of:-
	    DEFAULT_MAX_GROUP_LENGTH if not configured. It is passed as an extra parameter to the GroupSequenceTask.
	  </td></tr></table></td></tr></table></div>

    \section{Planetarium Operations}

    <description>
      Planetarium operations are performed during pre-arranged slots/windows setup in the window schedule file.
      This file is generated by the FTOC and uploaded during the day several times a month. During POS mode
      the telescope is controlled via a web interface (RTI) communicating with the PCR using the PCP protocol 
      relayed on to the RCS using JSOs.
    </description><p/><hr color="blue"/><br/><span style="display:block;font-size:16pt;font-color:blue;background:#ccff22"><u/> -    POS/G2 Command migration</span><p/><description>
	Migration to G2 future command set, implementation and ongoing test program.
      </description><p/><div style="display:block;position:relative;left:20;background:#ff22cc">
      Notes associated with this workflow.
      <ul><font color="blue"/></ul></div><p/><div style="display:block;position:relative;left:20px;width:90%;background:#cc22cc">
      Changelog for this workflow.
      <table border="1" bgcolor="lightgrey"/></div><p/><div style="display:block;position:relative;left:20px;width:90%;background:#cc22cc">
     New grouped stuff by class
     <table bgcolor="#ccddee" frame="hsides" border="1"/></div><p/><hr color="blue"/><br/><span style="display:block;font-size:16pt;font-color:blue;background:#ccff22"><u/> -    POS/Mosaicing</span><p/><description>
	Implementation and testing of the mosaicing procedure
      </description><p/><div style="display:block;position:relative;left:20;background:#ff22cc">
      Notes associated with this workflow.
      <ul><font color="blue"/></ul></div><p/><div style="display:block;position:relative;left:20px;width:90%;background:#cc22cc">
      Changelog for this workflow.
      <table border="1" bgcolor="lightgrey"/></div><p/><div style="display:block;position:relative;left:20px;width:90%;background:#cc22cc">
     New grouped stuff by class
     <table bgcolor="#ccddee" frame="hsides" border="1"/></div><p/><hr color="blue"/><br/><span style="display:block;font-size:16pt;font-color:blue;background:#ccff22"><u/> -    POS/Abort responses</span><p/><description>
	Handling the responses to various abort reasons and propagating back to RTI.
      </description><p/><div style="display:block;position:relative;left:20;background:#ff22cc">
      Notes associated with this workflow.
      <ul><font color="blue"/></ul></div><p/><div style="display:block;position:relative;left:20px;width:90%;background:#cc22cc">
      Changelog for this workflow.
      <table border="1" bgcolor="lightgrey"/></div><p/><div style="display:block;position:relative;left:20px;width:90%;background:#cc22cc">
     New grouped stuff by class
     <table bgcolor="#ccddee" frame="hsides" border="1"/></div>

    \section{Target of Opportunity Operations}

    <description>
      The TOCS is responsible for control of the telescope on inception of alerts of various types. The controller
      is in fact an intelligent agent located on a nearby proxy - (soas to provide fast return of image data for
      processing). This operating mode typically has priority over all others.
    </description><p/><hr color="blue"/><br/><span style="display:block;font-size:16pt;font-color:blue;background:#ccff22"><u>WF_002</u> -    TOCS/General Protocol</span><p/><description>
	Handling general TOCS protocol issues      
      </description><p/><div style="display:block;position:relative;left:20;background:#ff22cc">
      Notes associated with this workflow.
      <ul><font color="blue"/></ul></div><p/><div style="display:block;position:relative;left:20px;width:90%;background:#cc22cc">
      Changelog for this workflow.
      <table border="1" bgcolor="lightgrey"><tr><td fgcolor="blue">TOCServerImpl</td><td fgcolor="blue">2005-03-17T11:55</td><td fgcolor="blue">Changed timeout block to allow PSI to wait on task completion</td></tr></table></div><p/><div style="display:block;position:relative;left:20px;width:90%;background:#cc22cc">
     New grouped stuff by class
     <table bgcolor="#ccddee" frame="hsides" border="1"><tr><td valign="top"><b>TOCServerImpl</b><p/>ngat.rcs.tocs</td><td><table><tr><td valign="top">2005-03-17T11:55</td><td>Changed timeout block to allow PSI to wait on task completion</td></tr></table></td></tr></table></div><p/><hr color="blue"/><br/><span style="display:block;font-size:16pt;font-color:blue;background:#ccff22"><u>WF_003</u> -    TOCS/Instrumentation</span><p/><description>           
	This workflow is intended to cover any instrument related issues including 
	<ul>
	  <li>Enhancements to the INSTR command for new or existing instruments.</li>
	  <li>Testing of all instrument interactions</li>
	  <li>Task implementation relating to instruments</li>
	</ul>
      </description><p/><div style="display:block;position:relative;left:20;background:#ff22cc">
      Notes associated with this workflow.
      <ul><font color="blue"/></ul></div><p/><div style="display:block;position:relative;left:20px;width:90%;background:#cc22cc">
      Changelog for this workflow.
      <table border="1" bgcolor="lightgrey"><tr><td fgcolor="blue">TOC_GenericCommandImpl</td><td fgcolor="blue">2005-03-17T15:51</td><td fgcolor="blue">Added a new instrument command parameter for the IRCAM</td></tr><tr><td fgcolor="blue">TOC_GenericCommandImpl</td><td fgcolor="blue">2005-03-21T15:00</td><td fgcolor="blue">Added optional calibrate before and after parameters to (all) instrument commands.
	    If these are ommitted the default settings for the instrument are used.
	  </td></tr><tr><td fgcolor="blue">TOC_GenericCommandImpl</td><td fgcolor="blue">2006-09-06T12:04</td><td fgcolor="blue">Added options to INSTR command handling. New Instrument names
             
               RATCAM 
               As before but now specific for dev.lt.RATCamConfig
               HAWKCAM
               Generates a dev.lt.HawkCamConfig
               DILLCAM
                Intended to generate a dev.lt.DillCamConfig but does RATCam for now
             
          </td></tr></table></div><p/><div style="display:block;position:relative;left:20px;width:90%;background:#cc22cc">
     New grouped stuff by class
     <table bgcolor="#ccddee" frame="hsides" border="1"><tr><td valign="top"><b>TOC_GenericCommandImpl</b><p/>ngat.rcs.tocs</td><td><table><tr><td valign="top">2005-03-17T15:51</td><td>Added a new instrument command parameter for the IRCAM</td></tr><tr><td valign="top">2005-03-21T15:00</td><td>Added optional calibrate before and after parameters to (all) instrument commands.
	    If these are ommitted the default settings for the instrument are used.
	  </td></tr><tr><td valign="top">2006-09-06T12:04</td><td>Added options to INSTR command handling. New Instrument names
             
               RATCAM 
               As before but now specific for dev.lt.RATCamConfig
               HAWKCAM
               Generates a dev.lt.HawkCamConfig
               DILLCAM
                Intended to generate a dev.lt.DillCamConfig but does RATCam for now
             
          </td></tr></table></td></tr></table></div><p/><hr color="blue"/><br/><span style="display:block;font-size:16pt;font-color:blue;background:#ccff22"><u>WF_009</u> -    TOCS/Accounting</span><p/><description>           
	This workflow is intended to cover issues relating to TO accounting. This will include:-
	<ul>
	  <li>Update of Phase2DB with proposal/tag used time for the assigned TO proposal</li>
	  <li>TO Service accounting</li>
	</ul>
      </description><p/><div style="display:block;position:relative;left:20;background:#ff22cc">
      Notes associated with this workflow.
      <ul><font color="blue"/></ul></div><p/><div style="display:block;position:relative;left:20px;width:90%;background:#cc22cc">
      Changelog for this workflow.
      <table border="1" bgcolor="lightgrey"/></div><p/><div style="display:block;position:relative;left:20px;width:90%;background:#cc22cc">
     New grouped stuff by class
     <table bgcolor="#ccddee" frame="hsides" border="1"/></div>

    \section{Communications}

    <description>
      The RCS has to communicate with a number of external subsystems to instigate control of
      the telescope and its associated instruments and to collate environmental data. 
    </description><p/><hr color="blue"/><br/><span style="display:block;font-size:16pt;font-color:blue;background:#ccff22"><u>WF_001</u> -    COMMS/Command Translator</span><p/><description>
	The CommandTranslator is a module which translates commands received from the various
	executive tasks and status collators as JMS objects (JSOs) into the native format for
	the TCS and other external subsystems. Additionally native responses are translated
	back to JSOs to supply the appropriate type of response to the originating task/collator.             
      </description><p/><div style="display:block;position:relative;left:20;background:#ff22cc">
      Notes associated with this workflow.
      <ul><font color="blue"/></ul></div><p/><div style="display:block;position:relative;left:20px;width:90%;background:#cc22cc">
      Changelog for this workflow.
      <table border="1" bgcolor="lightgrey"><tr><td fgcolor="blue">LT_RGO_TCS_CommandTranslator</td><td fgcolor="blue">2005-03-17T08:00</td><td fgcolor="blue">Modified the translator for AZIMUTH command to convert from
	    encapsulated units (radians) to TCS units (degrees)</td></tr></table></div><p/><div style="display:block;position:relative;left:20px;width:90%;background:#cc22cc">
     New grouped stuff by class
     <table bgcolor="#ccddee" frame="hsides" border="1"><tr><td valign="top"><b>LT_RGO_TCS_CommandTranslator</b><p/>ngat.rcs.control</td><td><table><tr><td valign="top">2005-03-17T08:00</td><td>Modified the translator for AZIMUTH command to convert from
	    encapsulated units (radians) to TCS units (degrees)</td></tr></table></td></tr></table></div><p/><hr color="blue"/><br/><span style="display:block;font-size:16pt;font-color:blue;background:#ccff22"><u>WF_014</u> -    COMMS/CIL Handling</span><p/><description>
	The CIL layer is responsible for handling the sending and receiving of UDP messages via the
	CIL interface. Commands (JSO) are received from clients (TMM Tasks and SCM Collators) via a
	SlotConnection. Handlers are created and registered for these agianst a generated sequence number.
	The Reader thread receives replies and places into a queue. The despatcher checks the queue,
	looks up the registered handler and sends the translated reply to the original client.
      </description><p/><div style="display:block;position:relative;left:20;background:#ff22cc">
      Notes associated with this workflow.
      <ul><font color="blue"><li><u>ENT</u>, 2005-04-01T08:30 -  <note class="ENT" date="2005-04-01T08:30">
	The CIL handling has blown up several times now with the old - 'flat-line" effect i.e. the status
	informaiton is not updated. This is due to the Despatcher not proceeeding with the sending of the
	reply - generally an ERROR class reply. It now looks like this occurs with the following sequence:-

	<ul>
	  <li> Despatcher     -&gt; handler.proxyError(data)</li>
	  <li> Handler        -&gt; serverImpl.sendDone(data)</li>
	  <li> JMSServerImpl  -&gt; connection.send(done)</li>
	  <li> SlotConnection -&gt; send(done)</li>
	</ul>

	It looks like the call to SlotConnection.send() is blocked due to either:-
	<ul>
	  <li> Buffer is full (it is a single item buffer) - i.e. client has not emptied it.</li>
	  <li> Client failed to call notify on emptying.</li>
	</ul>

      </note></li></font></ul></div><p/><div style="display:block;position:relative;left:20px;width:90%;background:#cc22cc">
      Changelog for this workflow.
      <table border="1" bgcolor="lightgrey"><tr><td fgcolor="blue">RCSSubsystemConnectionFactory</td><td fgcolor="blue">2005-04-01T12:00</td><td fgcolor="blue">
	    Changed the default size for a SlotConnection to be 5 rather than 1.
	    This is hoped to avoid a full buffer even if there is something in it.
	  </td></tr><tr><td fgcolor="blue">JMSMAprotocolServerImpl</td><td fgcolor="blue">2005-04-01T12:00</td><td fgcolor="blue">
	    Modified the call to sendDone() to call iconnection.send(data, timeout)
	    using the value of handler.getHandlingTime().
	  </td></tr><tr><td fgcolor="blue">IConnection</td><td fgcolor="blue">2005-04-01T12:00</td><td fgcolor="blue">
	    Added a new abstract method send(data, timeout) to allow the send operation
	    to timeout. Currently only a receive can timeout.
	  </td></tr><tr><td fgcolor="blue">SlotConnection</td><td fgcolor="blue">2005-04-01T12:00</td><td fgcolor="blue">
	    Implemented the  new method send(data, timeout) to allow the send operation
	    to timeout.
	  </td></tr></table></div><p/><div style="display:block;position:relative;left:20px;width:90%;background:#cc22cc">
     New grouped stuff by class
     <table bgcolor="#ccddee" frame="hsides" border="1"><tr><td valign="top"><b>RCSSubsystemConnectionFactory</b><p/>ngat.rcs.comms</td><td><table><tr><td valign="top">2005-04-01T12:00</td><td>
	    Changed the default size for a SlotConnection to be 5 rather than 1.
	    This is hoped to avoid a full buffer even if there is something in it.
	  </td></tr></table></td></tr><tr><td valign="top"><b>JMSMAprotocolServerImpl</b><p/>ngat.net</td><td><table><tr><td valign="top">2005-04-01T12:00</td><td>
	    Modified the call to sendDone() to call iconnection.send(data, timeout)
	    using the value of handler.getHandlingTime().
	  </td></tr></table></td></tr><tr><td valign="top"><b>IConnection</b><p/>ngat.net</td><td><table><tr><td valign="top">2005-04-01T12:00</td><td>
	    Added a new abstract method send(data, timeout) to allow the send operation
	    to timeout. Currently only a receive can timeout.
	  </td></tr></table></td></tr><tr><td valign="top"><b>SlotConnection</b><p/>ngat.net</td><td><table><tr><td valign="top">2005-04-01T12:00</td><td>
	    Implemented the  new method send(data, timeout) to allow the send operation
	    to timeout.
	  </td></tr></table></td></tr></table></div>

    \section{Event Messaging}

    <description>
      Events are propagated around the system between the SCM, TMM and SM. These include:-
      
      <ul>
	<li> action events from the SM to initiate activity from the TMM.</li>
	<li> environmental alerts from the SCM to SM to prompt reactive behaviour.</li>
	<li> task events from the TMM to the SM to notify completion status.</li>
	<li> other control events can be generated by external users and other modules as required.</li>
      </ul>
    </description><p/><hr color="blue"/><br/><span style="display:block;font-size:16pt;font-color:blue;background:#ccff22"><u/> -    EMM/General implementation</span><p/><description>
	Implementation and testing of EMM.
	</description><p/><div style="display:block;position:relative;left:20;background:#ff22cc">
      Notes associated with this workflow.
      <ul><font color="blue"/></ul></div><p/><div style="display:block;position:relative;left:20px;width:90%;background:#cc22cc">
      Changelog for this workflow.
      <table border="1" bgcolor="lightgrey"/></div><p/><div style="display:block;position:relative;left:20px;width:90%;background:#cc22cc">
     New grouped stuff by class
     <table bgcolor="#ccddee" frame="hsides" border="1"/></div>

    \section{Instrument Support}

    <description> 
      The RCS must store details of the various Instrument Control Systems (ICS) and their
      ranges of available and current configurations. Ths ICS are able to send commands to
      the RCS for information and for forwarding (possibly modified) to the TCS in order to
      allow for instrument-specific positioning adjustments. This Workflow is intended to
      handle these issues.
    </description><p/><hr color="blue"/><br/><span style="display:block;font-size:16pt;font-color:blue;background:#ccff22"><u>WF_010</u> -    ISS/Instrument control actions</span><p/><description>
	This workflow deals with issues relating to the selection and configuration of
	instruments in Science,Planetarium and TO operations.
      </description><p/><div style="display:block;position:relative;left:20;background:#ff22cc">
      Notes associated with this workflow.
      <ul><font color="blue"/></ul></div><p/><div style="display:block;position:relative;left:20px;width:90%;background:#cc22cc">
      Changelog for this workflow.
      <table border="1" bgcolor="lightgrey"/></div><p/><div style="display:block;position:relative;left:20px;width:90%;background:#cc22cc">
     New grouped stuff by class
     <table bgcolor="#ccddee" frame="hsides" border="1"/></div><p/><hr color="blue"/><br/><span style="display:block;font-size:16pt;font-color:blue;background:#ccff22"><u>WF_012</u> -    ISS/Instrument adjustment of telescope</span><p/><description>
	This workflow is to deal with issues relating to the instruments' adjustment of
	the telescope via INST_ISS commands via the RCS to the TCS.
      </description><p/><div style="display:block;position:relative;left:20;background:#ff22cc">
      Notes associated with this workflow.
      <ul><font color="blue"><li><u>ENT</u>, 2005-03-23T13:33 -  <note class="ENT" date="2005-03-23T13:33">
	At the Dev Mtg 22-03-2005 it was decided to add a response delay to the ISS_OFFSET_RADEC
	command soas to ensure the telescope had stopped moving before the completion reply is
	sent back to the calling instrument.
      </note></li></font></ul></div><p/><div style="display:block;position:relative;left:20px;width:90%;background:#cc22cc">
      Changelog for this workflow.
      <table border="1" bgcolor="lightgrey"><tr><td fgcolor="blue">ISS_CommandImpl</td><td fgcolor="blue">2005-03-23T13:30</td><td fgcolor="blue">A new parameter responseDelay is calculated when the class is instantiated using
	    the overridable method calculateResponseDelay(). This by default returns 0 (zero). The 
	    handleResponse() method has been doctored to include a sleep for this period after receiving the reply before
	  sending on to the caller (ICS). There is some danger here in that the handleReply() method is called by
	  the CCT created by the CommandImpl whereas the handleRequest() method was called by the SCT and is in a 
	  timeout waitfor which is not currently updatable - i.e. it does not keep re-entering if the response
	  has not yet been detected</td></tr><tr><td fgcolor="blue">ISS_OFFSET_RA_DECCommandImpl</td><td fgcolor="blue">2005-03-23T13:30</td><td fgcolor="blue">The method calculateResponseDelay() is overridden to return 10 secs.
	  </td></tr></table></div><p/><div style="display:block;position:relative;left:20px;width:90%;background:#cc22cc">
     New grouped stuff by class
     <table bgcolor="#ccddee" frame="hsides" border="1"><tr><td valign="top"><b>ISS_CommandImpl</b><p/>ngat.rcs.iss</td><td><table><tr><td valign="top">2005-03-23T13:30</td><td>A new parameter responseDelay is calculated when the class is instantiated using
	    the overridable method calculateResponseDelay(). This by default returns 0 (zero). The 
	    handleResponse() method has been doctored to include a sleep for this period after receiving the reply before
	  sending on to the caller (ICS). There is some danger here in that the handleReply() method is called by
	  the CCT created by the CommandImpl whereas the handleRequest() method was called by the SCT and is in a 
	  timeout waitfor which is not currently updatable - i.e. it does not keep re-entering if the response
	  has not yet been detected</td></tr></table></td></tr><tr><td valign="top"><b>ISS_OFFSET_RA_DECCommandImpl</b><p/>ngat.rcs.iss</td><td><table><tr><td valign="top">2005-03-23T13:30</td><td>The method calculateResponseDelay() is overridden to return 10 secs.
	  </td></tr></table></td></tr></table></div><p/><hr color="blue"/><br/><span style="display:block;font-size:16pt;font-color:blue;background:#ccff22"><u/> -    ISS/FITS headers</span><p/><description>
	Generation of FITS headers from TCS queries and from internal state.
      </description><p/><div style="display:block;position:relative;left:20;background:#ff22cc">
      Notes associated with this workflow.
      <ul><font color="blue"/></ul></div><p/><div style="display:block;position:relative;left:20px;width:90%;background:#cc22cc">
      Changelog for this workflow.
      <table border="1" bgcolor="lightgrey"><tr><td fgcolor="blue">FITS_HeaderInfo</td><td fgcolor="blue">2005-09-28T15:32</td><td fgcolor="blue">
	    A large number of new headers have been added to cover most of the data available
	    via TCS SHOW commands. These are now grouped together into blocks related to the
	    ngat.message.RCS_TCS.TCS_Status segment in which they reside. The card numbers
	    are also grouped into decades. Some new MOON parameters are now available.
	    The limits segment has not been included. The units parameters
	    have mostly not yet been filled in.
	  </td></tr><tr><td fgcolor="blue">ISS_GET_FITSCommandImpl</td><td fgcolor="blue">2005-09-28T15:32</td><td fgcolor="blue">
	    The new FITS header cards are now inlcuded into the properties sent back to the
	    ISS in the reply message.
	  </td></tr></table></div><p/><div style="display:block;position:relative;left:20px;width:90%;background:#cc22cc">
     New grouped stuff by class
     <table bgcolor="#ccddee" frame="hsides" border="1"><tr><td valign="top"><b>FITS_HeaderInfo</b><p/>ngat.rcs.iss</td><td><table><tr><td valign="top">2005-09-28T15:32</td><td>
	    A large number of new headers have been added to cover most of the data available
	    via TCS SHOW commands. These are now grouped together into blocks related to the
	    ngat.message.RCS_TCS.TCS_Status segment in which they reside. The card numbers
	    are also grouped into decades. Some new MOON parameters are now available.
	    The limits segment has not been included. The units parameters
	    have mostly not yet been filled in.
	  </td></tr></table></td></tr><tr><td valign="top"><b>ISS_GET_FITSCommandImpl</b><p/>ngat.rcs.iss</td><td><table><tr><td valign="top">2005-09-28T15:32</td><td>
	    The new FITS header cards are now inlcuded into the properties sent back to the
	    ISS in the reply message.
	  </td></tr></table></td></tr></table></div>

    \section{Task Management}

    <description>
      The TMM is responsible for instigating actions on request from the SM. It consists of a hierarchy
      of ManagerTasks reported to by lower level Executive tasks which carry out telescope and instrument
      control actions. 
      The following components are used:-
      <dl>
	<dt>TMMSequencer</dt>
	<dd>handles requests from CA on MCA yield</dd>
	<dt>TMMRecoveryRegistry</dt>
	<dd>Centralized point for collecting task recovery information</dd>
	<dt>TMMCOnfigRegistry</dt>
	<dd>Centralized point for collecting task configuration</dd>
      </dl>
    </description><p/><hr color="blue"/><br/><span style="display:block;font-size:16pt;font-color:blue;background:#ccff22"><u/> -    TMM/Sequencer</span><p/><description>
	The Sequencer handles requests from the CA when the controlling MCA yields after completing a job.
	Here we cover the general protocol and interaction with the SM and CA.
      </description><p/><div style="display:block;position:relative;left:20;background:#ff22cc">
      Notes associated with this workflow.
      <ul><font color="blue"/></ul></div><p/><div style="display:block;position:relative;left:20px;width:90%;background:#cc22cc">
      Changelog for this workflow.
      <table border="1" bgcolor="lightgrey"/></div><p/><div style="display:block;position:relative;left:20px;width:90%;background:#cc22cc">
     New grouped stuff by class
     <table bgcolor="#ccddee" frame="hsides" border="1"/></div><p/><hr color="blue"/><br/><span style="display:block;font-size:16pt;font-color:blue;background:#ccff22"><u/> -    TMM/Recovery Registry</span><p/><description>
	The recovery registry contains details in a simplified format to control the response of MTAs to
	failure of their subservient MTAs and ETAs. The response can be customized to suit varying daily
	handling of errors.
      </description><p/><div style="display:block;position:relative;left:20;background:#ff22cc">
      Notes associated with this workflow.
      <ul><font color="blue"/></ul></div><p/><div style="display:block;position:relative;left:20px;width:90%;background:#cc22cc">
      Changelog for this workflow.
      <table border="1" bgcolor="lightgrey"/></div><p/><div style="display:block;position:relative;left:20px;width:90%;background:#cc22cc">
     New grouped stuff by class
     <table bgcolor="#ccddee" frame="hsides" border="1"/></div><p/><hr color="blue"/><br/><span style="display:block;font-size:16pt;font-color:blue;background:#ccff22"><u/> -    TMM/Config Registry</span><p/><description>
	The config registry contains details of configuration parameters used by any MTA or ETA which needs
	to be set up from day to day to suit operational requirements.
      </description><p/><div style="display:block;position:relative;left:20;background:#ff22cc">
      Notes associated with this workflow.
      <ul><font color="blue"/></ul></div><p/><div style="display:block;position:relative;left:20px;width:90%;background:#cc22cc">
      Changelog for this workflow.
      <table border="1" bgcolor="lightgrey"/></div><p/><div style="display:block;position:relative;left:20px;width:90%;background:#cc22cc">
     New grouped stuff by class
     <table bgcolor="#ccddee" frame="hsides" border="1"/></div><p/><hr color="blue"/><br/><span style="display:block;font-size:16pt;font-color:blue;background:#ccff22"><u/> -    TMM/General Mode Control</span><p/><description>
      Covers general issues relating to mode control agents.
</description><p/><div style="display:block;position:relative;left:20;background:#ff22cc">
      Notes associated with this workflow.
      <ul><font color="blue"/></ul></div><p/><div style="display:block;position:relative;left:20px;width:90%;background:#cc22cc">
      Changelog for this workflow.
      <table border="1" bgcolor="lightgrey"/></div><p/><div style="display:block;position:relative;left:20px;width:90%;background:#cc22cc">
     New grouped stuff by class
     <table bgcolor="#ccddee" frame="hsides" border="1"/></div>

    \section{Status Collation}

    <description>
      The SCM is responsible for collecting environmental status from external systems and for
      filtering and applying rules to allow reactive input to the overall control model.
    </description><p/><hr color="blue"/><br/><span style="display:block;font-size:16pt;font-color:blue;background:#ccff22"><u>WF_005</u> -    SCM/Sensors</span><p/><description>
	Sensors are used to store and present data from various collation mechanisms using a standard
	inerface..
      </description><p/><div style="display:block;position:relative;left:20;background:#ff22cc">
      Notes associated with this workflow.
      <ul><font color="blue"/></ul></div><p/><div style="display:block;position:relative;left:20px;width:90%;background:#cc22cc">
      Changelog for this workflow.
      <table border="1" bgcolor="lightgrey"><tr><td fgcolor="blue">CCDSeeing</td><td fgcolor="blue">2005-03-23T12:32</td><td fgcolor="blue">Added a new temporary field lastSeeing to store the last value of the atmospheric.seeing
	    parameter which was used to update this sensor. This is to be used by input devices to allow
	    a form of prefiltering of the input data soas to disallow 'unbelievable' data.</td></tr><tr><td fgcolor="blue">ExposureTask</td><td fgcolor="blue">2005-03-23T12:32</td><td fgcolor="blue">Added new parameter max_seeing to record the maximum seeing we will accept
	    as being valid as returned from the DPRT in the DP_ACK message. This is configured from the
	    task.config as parameter max.usable.seeing under this task's id. If not set the default value supplied as
	    the constant DEFAULT_MAX_SEEING is used. When a new seeing value is obtained from the DPRT this is compared
	    to the max_seeing and if it is more, the value obtained from CCDSeeing.getLastSeeing() 
	    is used and fed back into the CCDSeing sensor. - this is a temporary fix and really needs to be handled 
	    by modification of the class used as the seeing predictor.
	  </td></tr></table></div><p/><div style="display:block;position:relative;left:20px;width:90%;background:#cc22cc">
     New grouped stuff by class
     <table bgcolor="#ccddee" frame="hsides" border="1"><tr><td valign="top"><b>CCDSeeing</b><p/>ngat.rcs.scm.collation</td><td><table><tr><td valign="top">2005-03-23T12:32</td><td>Added a new temporary field lastSeeing to store the last value of the atmospheric.seeing
	    parameter which was used to update this sensor. This is to be used by input devices to allow
	    a form of prefiltering of the input data soas to disallow 'unbelievable' data.</td></tr></table></td></tr><tr><td valign="top"><b>ExposureTask</b><p/>ngat.rcs.tmm.executive</td><td><table><tr><td valign="top">2005-03-23T12:32</td><td>Added new parameter max_seeing to record the maximum seeing we will accept
	    as being valid as returned from the DPRT in the DP_ACK message. This is configured from the
	    task.config as parameter max.usable.seeing under this task's id. If not set the default value supplied as
	    the constant DEFAULT_MAX_SEEING is used. When a new seeing value is obtained from the DPRT this is compared
	    to the max_seeing and if it is more, the value obtained from CCDSeeing.getLastSeeing() 
	    is used and fed back into the CCDSeing sensor. - this is a temporary fix and really needs to be handled 
	    by modification of the class used as the seeing predictor.
	  </td></tr></table></td></tr></table></div><p/><hr color="blue"/><br/><span style="display:block;font-size:16pt;font-color:blue;background:#ccff22"><u>WF_007</u> -    SCM/Filters</span><p/><description>
	Filters are used to smooth the output gathered from sensors. There are a number of different
	classes of filter.
      </description><p/><div style="display:block;position:relative;left:20;background:#ff22cc">
      Notes associated with this workflow.
      <ul><font color="blue"><li><u>TBD</u>, 2005-03-21T16:00 -  <note class="TBD" date="2005-03-21T16:00">Need to sort out the rain filter for LT which shows reverse state 
	from what it actually is
      </note></li><li><u>TBD</u>, 2005-03-22T07:52 -  <note class="TBD" date="2005-03-22T07:52"> 
	There is a general problem with modal and steady filters we cannot set these up using a time period 
	<p/>i.e. we want to be able to say something like 
	  - return the steady (good) value over 20 minutes OR display a default (bad) value   
      </note></li></font></ul></div><p/><div style="display:block;position:relative;left:20px;width:90%;background:#cc22cc">
      Changelog for this workflow.
      <table border="1" bgcolor="lightgrey"/></div><p/><div style="display:block;position:relative;left:20px;width:90%;background:#cc22cc">
     New grouped stuff by class
     <table bgcolor="#ccddee" frame="hsides" border="1"/></div><p/><hr color="blue"/><br/><span style="display:block;font-size:16pt;font-color:blue;background:#ccff22"><u>WF_008</u> -    SCM/Rules</span><p/><description>
	The rulebase contains details of how the output from filtered sensor readings is to be used to trigger
	events to be propageted to the SM and TMM.
      </description><p/><div style="display:block;position:relative;left:20;background:#ff22cc">
      Notes associated with this workflow.
      <ul><font color="blue"/></ul></div><p/><div style="display:block;position:relative;left:20px;width:90%;background:#cc22cc">
      Changelog for this workflow.
      <table border="1" bgcolor="lightgrey"/></div><p/><div style="display:block;position:relative;left:20px;width:90%;background:#cc22cc">
     New grouped stuff by class
     <table bgcolor="#ccddee" frame="hsides" border="1"/></div><p/><hr color="blue"/><br/><span style="display:block;font-size:16pt;font-color:blue;background:#ccff22"><u>WF_011</u> -    SCM/Monitoring</span><p/><description>
	General monitoring issues.
      </description><p/><div style="display:block;position:relative;left:20;background:#ff22cc">
      Notes associated with this workflow.
      <ul><font color="blue"><li><u>ENT</u>, 2005-03-29T10:45 -  <note class="ENT" date="2005-03-29T10:45">
	There is a general problem with the LT in that due to the non-wrapped pipes for the ratcam, the rotator
	is run in (ROT MOUNT 0 + ROT FLOAT) mode. This has resulted in a 'sticking' of the rotator during exposures.
	The TrackingMonitor class is an experimental stand-alone class designed to receive status updates and
	make decisions based on this data about the state of the rotator. This monitor requires to be enabled only
	when an exposure is underway - i.e. we do not want to be firing off aborts during slews etc.
      </note></li><li><u>OBS</u>, 2005-03-30T09:02 -  <note class="OBS" date="2005-03-30T09:02">
	Tested rotator exercise mode last night on LT. The rotator was unable to move to the requested angle
	of 20 degrees and caused timeouts on both the SLEW and ROT MOU commands - these were retried and failed
	again - rot-ex was disabled. The rotator tracking detection however appears to work OK - the rotator switched
	to moving state during observations and this was picked up and X_ROT_TRACK_LOST event fired at the correct time
	- note that it continues to be fired every update - not required but normally we would have aborted the current
	Observation task by then - there may be a problem however if this gets fired off before the abort has actually 
	started -needs an inhibit on the event after firing - i.e. we only ever need one tracking lost event to fire 
	during an observation.
      </note></li><li><u>ENT</u>, 2005-09-28T16:55 -  <note class="ENT" date="2005-09-28T16:55">
	The autoguider is to be deployed shortly for testing. These is a need to monitor the autiguider LOCK state
	while it is meant to be guiding a target. A new class AutoguiderMonitor related closely to the current
	TrackingMonitor is to be created to handle lost locking of AG. This monitor requires to be enabled only
	when an exposure is underway - i.e. we do not want to be firing off aborts during slews etc.
      </note></li><li><u>ENT</u>, 2005-09-28T16:55 -  <note class="ENT" date="2005-09-28T16:55">
	There may be some use in tying the 2 standalone monitors for Autoguider and Tracking into a more
	generally configurable class which can be configured in SCM - new base class would be required
	as well as appropriate and general enough configuration. Also a new mapping would be provided
	via SCM or Monitors classes to access these.
      </note></li></font></ul></div><p/><div style="display:block;position:relative;left:20px;width:90%;background:#cc22cc">
      Changelog for this workflow.
      <table border="1" bgcolor="lightgrey"><tr><td fgcolor="blue">ExposureTask</td><td fgcolor="blue">2005-03-29T10:45</td><td fgcolor="blue"> 
	  The TrackingMonitor is already switched on at the start of the exposure i.e. during the onInit() call
	  to .TrackingMonitor.reset().
	  Now calls TrackingMonitor.setEnableAlerts(false) after the exposure has completed - i.e. onDisposal
	  of this task.
	</td></tr><tr><td fgcolor="blue">RCSController</td><td fgcolor="blue">2005-03-29T10:45</td><td fgcolor="blue"> 
	  New config variable max.tracking.lost.time default (10 sec) defines the time in millis for which tracking can 
	  be lost before the TrackingMonitor signals an alert. 
	</td></tr><tr><td fgcolor="blue">TrackingMonitor</td><td fgcolor="blue">2005-03-29T10:45</td><td fgcolor="blue"> 
	  New variables rotTrackIsLost rotTrackLostAt rotTrackLostFor alertsEnabled to determine when and
	  if the rotator lost tracking during an exposure and whether to signal an alert. 
	  New method enableAlerts() is called on reset to allow the posting of X_ROT_TRACK_LOST event
	  when the rotator is seen to be not tracking for greater than maxTrackLostTime. 
	  Some debug message code is generate by this class to indicate tracking status etc.
	</td></tr><tr><td fgcolor="blue">TrackingMonitor</td><td fgcolor="blue">2005-03-30T09:57</td><td fgcolor="blue"> 
	    Modified soas to disable alerts after firing off an alert event. The alerts will then only be re-enabled
	    by a reset when a new exposure is started.
	  </td></tr><tr><td fgcolor="blue">AutoguiderMonitor</td><td fgcolor="blue">2005-09-28T16:55</td><td fgcolor="blue"> 
	    New class to handle lost guiding of AG. This class is enabled by calls to reset()
	  </td></tr><tr><td fgcolor="blue">ExposureTask</td><td fgcolor="blue">2005-09-28T16:55</td><td fgcolor="blue"> 
	    Added call to AutoguiderMonitor.reset() during onInit().
	    Calls AutoguiderMonitor.setEnableAlerts(false) after the exposure has completed - 
	    i.e. onDisposal() of this task.
	  </td></tr><tr><td fgcolor="blue">RCSController</td><td fgcolor="blue">2005-09-28T16:55</td><td fgcolor="blue"> 
	    New config variable max.guiding.lost.time default (10 sec) defines the time in millis for which guiding can 
	    be lost before the AutoguiderMonitor signals an alert. 
	  </td></tr></table></div><p/><div style="display:block;position:relative;left:20px;width:90%;background:#cc22cc">
     New grouped stuff by class
     <table bgcolor="#ccddee" frame="hsides" border="1"><tr><td valign="top"><b>ExposureTask</b><p/>ngat.rcs.tmm.executive</td><td><table><tr><td valign="top">2005-03-29T10:45</td><td> 
	  The TrackingMonitor is already switched on at the start of the exposure i.e. during the onInit() call
	  to .TrackingMonitor.reset().
	  Now calls TrackingMonitor.setEnableAlerts(false) after the exposure has completed - i.e. onDisposal
	  of this task.
	</td></tr><tr><td valign="top">2005-09-28T16:55</td><td> 
	    Added call to AutoguiderMonitor.reset() during onInit().
	    Calls AutoguiderMonitor.setEnableAlerts(false) after the exposure has completed - 
	    i.e. onDisposal() of this task.
	  </td></tr></table></td></tr><tr><td valign="top"><b>RCSController</b><p/>ngat.rcs</td><td><table><tr><td valign="top">2005-03-29T10:45</td><td> 
	  New config variable max.tracking.lost.time default (10 sec) defines the time in millis for which tracking can 
	  be lost before the TrackingMonitor signals an alert. 
	</td></tr><tr><td valign="top">2005-09-28T16:55</td><td> 
	    New config variable max.guiding.lost.time default (10 sec) defines the time in millis for which guiding can 
	    be lost before the AutoguiderMonitor signals an alert. 
	  </td></tr></table></td></tr><tr><td valign="top"><b>TrackingMonitor</b><p/>ngat.rcs.scm.detection</td><td><table><tr><td valign="top">2005-03-29T10:45</td><td> 
	  New variables rotTrackIsLost rotTrackLostAt rotTrackLostFor alertsEnabled to determine when and
	  if the rotator lost tracking during an exposure and whether to signal an alert. 
	  New method enableAlerts() is called on reset to allow the posting of X_ROT_TRACK_LOST event
	  when the rotator is seen to be not tracking for greater than maxTrackLostTime. 
	  Some debug message code is generate by this class to indicate tracking status etc.
	</td></tr><tr><td valign="top">2005-03-30T09:57</td><td> 
	    Modified soas to disable alerts after firing off an alert event. The alerts will then only be re-enabled
	    by a reset when a new exposure is started.
	  </td></tr></table></td></tr><tr><td valign="top"><b>AutoguiderMonitor</b><p/>ngat.rcs.scm.detection</td><td><table><tr><td valign="top">2005-09-28T16:55</td><td> 
	    New class to handle lost guiding of AG. This class is enabled by calls to reset()
	  </td></tr></table></td></tr></table></div>

    \section{State Decision Model}

    <description>
      The State Decision Model (SM) provides a link between the reactive inputs and the task actions
    </description><p/><hr color="blue"/><br/><span style="display:block;font-size:16pt;font-color:blue;background:#ccff22"><u>WF_016</u> -    STATEMODEL/Changes to SM</span><p/><description>
	This workflow covers the SM changes and upgrades
      </description><p/><div style="display:block;position:relative;left:20;background:#ff22cc">
      Notes associated with this workflow.
      <ul><font color="blue"/></ul></div><p/><div style="display:block;position:relative;left:20px;width:90%;background:#cc22cc">
      Changelog for this workflow.
      <table border="1" bgcolor="lightgrey"><tr><td fgcolor="blue">RCSControlAgent</td><td fgcolor="blue">2005-04-05T12:47</td><td fgcolor="blue">
	    A new action MAKE_SAFE is added with associated actionImplementor safeAction().
	    This causes a MakeSafeTask to be instantiated and fires off the SAFE_DONE or SAFE_FAILED
	    events on completion.
	  </td></tr><tr><td fgcolor="blue">MakeSafeTask</td><td fgcolor="blue">2005-04-05T12:47</td><td fgcolor="blue">
	    New task to implement the SAFE action. This fires off TRACK OF on the 3 mech axes in succession
	    with 5 second delays. Followed by PARK at 25 secs. If any task fails it is skipped except for PARK
	    which has 2-3 goes.
	  </td></tr><tr><td fgcolor="blue"/><td fgcolor="blue">2005-04-05T12:47</td><td fgcolor="blue">
	    Added new variable SAFE_OP to record the status of the MAKE_SAFE action.	    
	</td></tr><tr><td fgcolor="blue"/><td fgcolor="blue">2005-04-05T12:47</td><td fgcolor="blue">
	    New transient state TRANS_SAFE entered after ABORT_NIGHT_OPS with execution of the MAKE_SAFE
	    action. SAFE_DONE or SAFE_FAILED move the SM onto INHIB_OPS. 
	</td></tr></table></div><p/><div style="display:block;position:relative;left:20px;width:90%;background:#cc22cc">
     New grouped stuff by class
     <table bgcolor="#ccddee" frame="hsides" border="1"><tr><td valign="top"><b>RCSControlAgent</b><p/>ngat.rcs.tmm.manager</td><td><table><tr><td valign="top">2005-04-05T12:47</td><td>
	    A new action MAKE_SAFE is added with associated actionImplementor safeAction().
	    This causes a MakeSafeTask to be instantiated and fires off the SAFE_DONE or SAFE_FAILED
	    events on completion.
	  </td></tr></table></td></tr><tr><td valign="top"><b>MakeSafeTask</b><p/>ngat.rcs.tmm.manager</td><td><table><tr><td valign="top">2005-04-05T12:47</td><td>
	    New task to implement the SAFE action. This fires off TRACK OF on the 3 mech axes in succession
	    with 5 second delays. Followed by PARK at 25 secs. If any task fails it is skipped except for PARK
	    which has 2-3 goes.
	  </td></tr></table></td></tr></table></div>

    \section{Control and monitoring}

    <description>
      This project is concerned with interactions between the RCS and external GUI clients - both as controllers
      and as receivers of telemetry.
    </description><p/><hr color="blue"/><br/><span style="display:block;font-size:16pt;font-color:blue;background:#ccff22"><u/> -    CONTROL/GUI control</span><p/><description>
	Covers the commands sent by the GUI and their implementation at the RCS.
	</description><p/><div style="display:block;position:relative;left:20;background:#ff22cc">
      Notes associated with this workflow.
      <ul><font color="blue"/></ul></div><p/><div style="display:block;position:relative;left:20px;width:90%;background:#cc22cc">
      Changelog for this workflow.
      <table border="1" bgcolor="lightgrey"/></div><p/><div style="display:block;position:relative;left:20px;width:90%;background:#cc22cc">
     New grouped stuff by class
     <table bgcolor="#ccddee" frame="hsides" border="1"/></div><p/><hr color="blue"/><br/><span style="display:block;font-size:16pt;font-color:blue;background:#ccff22"><u/> -    CONTROL/Monitoring</span><p/><description>
	Covers the data sent by the RCS to the GUI or other data handlers.
      </description><p/><div style="display:block;position:relative;left:20;background:#ff22cc">
      Notes associated with this workflow.
      <ul><font color="blue"/></ul></div><p/><div style="display:block;position:relative;left:20px;width:90%;background:#cc22cc">
      Changelog for this workflow.
      <table border="1" bgcolor="lightgrey"/></div><p/><div style="display:block;position:relative;left:20px;width:90%;background:#cc22cc">
     New grouped stuff by class
     <table bgcolor="#ccddee" frame="hsides" border="1"/></div>

    \section{Configuration}

    <description>
      The RCS has a large number of configuration files - this project aims to keep track of modifications
      and upgrades to these.
    </description>

    \section{Logging}

    <description>
      This project aims to keep track of any changes made to the RCS logging facilities.
    </description><p/><hr color="blue"/><br/><span style="display:block;font-size:16pt;font-color:blue;background:#ccff22"><u>WF_004</u> -    LOGGING/MCA Specific logging</span><p/><description>      
	This workflow is aimed at logging for the various control agents.
      </description><p/><div style="display:block;position:relative;left:20;background:#ff22cc">
      Notes associated with this workflow.
      <ul><font color="blue"/></ul></div><p/><div style="display:block;position:relative;left:20px;width:90%;background:#cc22cc">
      Changelog for this workflow.
      <table border="1" bgcolor="lightgrey"/></div><p/><div style="display:block;position:relative;left:20px;width:90%;background:#cc22cc">
     New grouped stuff by class
     <table bgcolor="#ccddee" frame="hsides" border="1"/></div><p/><hr color="blue"/><br/><span style="display:block;font-size:16pt;font-color:blue;background:#ccff22"><u>WF_006</u> -    LOGGING/Task logging</span><p/><description>
	This workflow is aimed at the task (TMM) module logging.
      </description><p/><div style="display:block;position:relative;left:20;background:#ff22cc">
      Notes associated with this workflow.
      <ul><font color="blue"/></ul></div><p/><div style="display:block;position:relative;left:20px;width:90%;background:#cc22cc">
      Changelog for this workflow.
      <table border="1" bgcolor="lightgrey"/></div><p/><div style="display:block;position:relative;left:20px;width:90%;background:#cc22cc">
     New grouped stuff by class
     <table bgcolor="#ccddee" frame="hsides" border="1"/></div><p/><hr color="blue"/><br/><span style="display:block;font-size:16pt;font-color:blue;background:#ccff22"><u>WF_014</u> -    LOGGING/Remote logging</span><p/><description>
	This workflow is associated with remote logging - i.e. from a GUI or external client.
     </description><p/><div style="display:block;position:relative;left:20;background:#ff22cc">
      Notes associated with this workflow.
      <ul><font color="blue"/></ul></div><p/><div style="display:block;position:relative;left:20px;width:90%;background:#cc22cc">
      Changelog for this workflow.
      <table border="1" bgcolor="lightgrey"><tr><td fgcolor="blue">GRAB_LOGImpl</td><td fgcolor="blue">2005-03-30T14:08</td><td fgcolor="blue">
	    Changed this class soas to add a new TelemetryLogHandler to the named logger. This should send
	    back each valid log message as a string formatted in a specific way (defaults to BasicLogFormatter).
	  </td></tr><tr><td fgcolor="blue">TelemetryLogHandler</td><td fgcolor="blue">2005-03-30T14:08</td><td fgcolor="blue">
	      New LogHandler to push LogRecords to the Telemetry sender. Just publishes formatted log records
	      as strings with appropriate formatting.
	    </td></tr></table></div><p/><div style="display:block;position:relative;left:20px;width:90%;background:#cc22cc">
     New grouped stuff by class
     <table bgcolor="#ccddee" frame="hsides" border="1"><tr><td valign="top"><b>GRAB_LOGImpl</b><p/>ngat.rcs.control</td><td><table><tr><td valign="top">2005-03-30T14:08</td><td>
	    Changed this class soas to add a new TelemetryLogHandler to the named logger. This should send
	    back each valid log message as a string formatted in a specific way (defaults to BasicLogFormatter).
	  </td></tr></table></td></tr><tr><td valign="top"><b>TelemetryLogHandler</b><p/>ngat.rcs.control</td><td><table><tr><td valign="top">2005-03-30T14:08</td><td>
	      New LogHandler to push LogRecords to the Telemetry sender. Just publishes formatted log records
	      as strings with appropriate formatting.
	    </td></tr></table></td></tr></table></div>